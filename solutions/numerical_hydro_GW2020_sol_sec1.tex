\documentclass[10pt]{article}
\usepackage{titlesec}
\usepackage{titling}
\usepackage[dvipsnames]{xcolor}
\usepackage[colorlinks=true, urlcolor=MidnightBlue, linkcolor=Magenta]{hyperref}
\usepackage{amsmath}
\usepackage[margin=1in]{geometry}
\usepackage{libertine}
\usepackage{lastpage}
\usepackage{fancyhdr}
\pagestyle{fancy}
\fancyhf{}
\fancyhead[LE,RO]{\leftmark}
\fancyhead[RE,LO]{\rightmark}
\fancyfoot[RE,LO]{\itshape ICTS GW Summer School 2020: Solutions of Numerical Hydrodynamics Exercises}
\fancyfoot[LE,RO]{\thepage/\pageref{LastPage}}
\renewcommand{\footrulewidth}{1pt}

\titleformat{\section}
{\large \bfseries}
{}
{0em}
{}[\titlerule]

\titleformat{\subsection}[hang]
{\bfseries}
{}
{0em}
{}

\makeatletter
\newcommand\institute[1]{\renewcommand\@institute{#1}}
\newcommand\@institute{}


\newcommand\group[1]{\renewcommand\@group{#1}}
\newcommand\@group{}


\newcommand\period[1]{\renewcommand\@period{#1}}
\newcommand\@period{}


\newcommand\course[1]{\renewcommand\@course{#1}}
\newcommand\@course{}

\newcommand\instructor[1]{\renewcommand\@instructor{#1}}
\newcommand\@instructor{}

\newcommand\email[1]{\renewcommand\@email{#1}}
\newcommand\@email{}

\renewcommand{\maketitle}{%
  \begin{flushleft}
    {\scshape \Large \bfseries \@title}\\
    {\@course,\@period}

    \vspace{0.1cm}

    {Instructor: \@instructor}

    \vspace{0.2cm}
    
    {\@author, \@group, \href{mailto:\@email}{\@email}}
    
    \vspace{0.1cm}
    
    {\@institute}
    \rule{\textwidth}{1pt}
  \end{flushleft}
}
\makeatother


\begin{document}
\title{Solutions of Numerical Hydrodynamics Exercises}
\author{Md Arif Shaikh}
\email{arif.shaikh@icts.res.in}
\group{Astrophysical Relatvity}
\institute{\href{https://www.icts.res.in}{International Centre for Theoretical Sciences}, Bengaluru, India.}
\course{Gravitational Wave Astrophysics Summer School at ICTS-TIFR}
\instructor{Ian Hawke, STAG, University of Southampton}
\period{18--22 May, 2020}
\maketitle

\tableofcontents
\thispagestyle{empty}

\section{Numerical Methods}
Find the problem set \href{https://drive.google.com/file/d/1nc5vWHYxZykUQVLmGtItXKbc399mgRxj/view}{here}.
\label{sec:numerical-methods}

\subsection{The conservation laws}
\label{sec:conservation-laws}

\begin{enumerate}
\item The metric can be written as $ds^2 = g_{a b}dx^a dx^b = e^{(m)}_a e^{(n)}_b g_{mn}dx^a dx^b = e^{(m)} e^{(n)} g_{mn}$, where $e^{(m)} = e^{(m)}_a dx^a$. The tetrad is orhthonormal if $g_{mn} = \eta_{mn}$ which implies $ds^2 = e^{(m)} e^{(n)} \eta_{mn}$. Thus we can easily read off the $e^{(m)}$ from the line element $ds^2$.
  \begin{itemize}
  \item \underline{Orhthonormal tetrad for Minkowski spacetime in Cartesian coordinates:}
    The metric has the form
    \begin{equation}
      \label{eq:minkowski-cartesian}
      ds^2 = -dt^2 + dx^2 + dy^2 + dz^2
    \end{equation}
    We can easily read off the followings $e^{(0)} = dt, e^{(1)} = dx, e^{(2)} = dy, e^{(3)} = dz$. And hence, using $e^{(m)} = e^{(m)}_a dx^a$, we find
    \begin{equation}
      \label{eq:orthonormal-tetrad-minkowski-cartesian}
      \boxed{e^{(0)}_a = \delta^t_a, e^{(1)}_a = \delta^x_a, e^{(2)}_a = \delta^y_a, e^{(3)}_a = \delta^z_a}.
    \end{equation}

  \item \underline{Orhthonormal tetrad for Minkowski spacetime in spherical coordinates:} The metric has the form
    \begin{equation}
      \label{eq:minkowski-spherical}
      ds^2 = -dt^2 + dr^2 + r^2(d\theta^2 + \sin^2\theta d\phi^2).
    \end{equation}
    So $e^{(0)} = dt, e^{(1)} = dr, e^{(2)} = rd\theta, e^{(3)} = r\sin\theta d\phi$ and
    \begin{equation}
      \label{eq:orthonormal-tetrad-minkowski-spherical}
      \boxed{e^{(0)}_a = \delta^t_a, e^{(1)}_a = \delta^r_a, e^{(2)}_a = r \delta^\theta_a, e^{(3)}_a = r\sin\theta \delta^\phi_a}.
    \end{equation}
  \end{itemize}
\item We have the identity
  \begin{equation}
    \label{eq:identity-1}
    \nabla_a V^a = \frac{1}{\sqrt{-g}}\partial_a\left(\sqrt{-g}V^a\right)
  \end{equation}
  and the conservation law $\nabla_a T^a_b=0$. Using these two, we get
  \begin{equation}
    \label{eq:1.2-a}
    \nabla_a\left(T^a_b W^b\right) = T^a_b \nabla_aW^b = \frac{1}{\sqrt{-g}} \partial_a (\sqrt{-g}T^a_b W^b).
  \end{equation}
  Thus,
  \begin{equation}
    \label{eq:identity-2}
    \boxed{\partial_a(\sqrt{-g}T^a_b W^b) = \sqrt{-g}T^a_b \nabla_a W^b}.
  \end{equation}
\item We choose an orthonormal tetrad $e_{(m)} = \{\partial_t, \partial_j\}$, i.e., $e_{(0)} = \partial_t, e_{(1)} = \partial_x, e_{(2)} = \partial_y, e_{(2)} = \partial_z$. Now we can expand $e_{(m)}$ in the coordinate basis as $e_{(m)} = e^a_{(m)} \partial_a$. So $e^a_{(m)} = \delta^{a}_m$ and hence $\partial_b e^a_{(m)} = 0$ which implies $\nabla_a e^b_{(m)} = \Gamma^b_{ac} e^c_{(m)}$. Setting $W^b = e^b_{(m)}$ in Eq. \ref{eq:identity-2} gives
  \begin{equation}
    \label{eq:1.2-c}
    \boxed{\partial_a(\sqrt{-g}T^a_b e^b_{(m)}) = \sqrt{-g}T^a_b \nabla_a e^b_{(m)} = \sqrt{-g}T^a_b \Gamma^b_{ac}e^c_{(m)}}.
  \end{equation}
\item Using $m=0$ and  $e^a_{(m)} = \delta^a_m$ in the r.h.s of Eq. (\ref{eq:1.2-c}) gives
  \begin{equation}
    \label{eq:t-comp-consv}
    \partial_a (\sqrt{-g}T^a_b \delta^{b}_0) = \partial_a (\sqrt{-g}T^a_t) = \partial_t (\sqrt{-g} T^t_t) + \partial_k (\sqrt{-g}T^k_t)
  \end{equation}
  and the l.h.s gives
  \begin{equation}
    \label{eq:1}
    \sqrt{-g}T^a_b \Gamma^b_{ac}\delta^c_0 = \sqrt{-g}T^a_b \Gamma^b_{at}
  \end{equation}
  which gives
  \begin{equation}
    \label{eq:m=0}
    \boxed{\partial_t (\sqrt{-g} T^t_t) + \partial_k (\sqrt{-g}T^k_t) = \sqrt{-g}T^a_b \Gamma^b_{at}}.
  \end{equation}
  and similarly for the spatial components i.e., $m=j$,
  \begin{equation}
    \label{eq:m=j}
    \boxed{\partial_t (\sqrt{-g} T^t_j) + \partial_k (\sqrt{-g}T^k_j) = \sqrt{-g}T^a_b \Gamma^b_{aj}}.
  \end{equation}
  Now, in Cartesian coordinates, $\sqrt{-g} = 1$ and  $\Gamma$s are zero. Hence the above two equations could be written in the following form
  \begin{equation}
    \label{eq:conservation-cartesian}
    \boxed{\partial_t\mathbf{q} + \partial_k \mathbf{f}^k(\mathbf{q}) = 0}
  \end{equation}
  where $\mathbf{q} = (T^t_t, T^t_j)^T$ and $\mathbf{f} = (T^k_t, T^k_j)^T$.
\item In spherical coordinates the $\Gamma$s are non-zero and also $e_{(m)}^b$ may not be constant. These result in the source terms. Also $\sqrt{-g} = r^2 \sin\theta \neq 1$. Thus the conservation equation has the following form
  \begin{equation}
    \label{eq:conservation-spherical}
    \boxed{\partial_t \mathbf{q} + \partial_k \mathbf{f}^k(\mathbf{q}) = s(\mathbf{q})}
  \end{equation}
  where $\mathbf{q} = (\sqrt{-g}T^t_t, \sqrt{-g}T^t_j)^T$, $\mathbf{f} = (\sqrt{-g}T^k_t, \sqrt{-g}T^k_j)^T$ and $s = \sqrt{-g}T^a_b(\Gamma^b_{at}, \Gamma^b_{a c}e^c_{(j)} + \partial_ae^b_{(j)})^T$ which clearly does not contain any derivative of the matter fields.
\end{enumerate}

\subsection{The perfect fluid}
\label{sec:perfect-fluid}
\begin{enumerate}
\item For perfect fluid the stress-energy tensor is given by
  \begin{equation}
    \label{eq:stress-energy-perfect-fluid}
    T_{ab} = \rho_0 h u_a u_b + p g_{ab}. 
  \end{equation}
  Enthalpy is defined by
  \begin{equation}
    \label{eq:enthalpy}
    h = 1 + \epsilon + \frac{p}{\rho_0}.
  \end{equation}
  and the equation of state is given by 
  \begin{equation}
    \label{eq:equation-of-state}
    p = p(\rho_0, \epsilon).
  \end{equation}
  Now considering constraint $u^au_a = -1$, we have three independent velocity components and independent variables $\rho_0$ and $\epsilon$ and hence total five unknown independent variables. Therefore there are five degrees of freedom.

\item $n^a = \alpha^{-1}(1, -\beta^i)$, where $\alpha, \beta$ are the lapse and shift and $n^a$ is the unit normal to the spatial slices. We split a four velocity into a piece along $n^a$ and a piece orthogonal to it,
  \begin{equation}
    \label{eq:3+1-decomposition}
    u_a = W(n_a + v_a),\quad n^av_a = 0.
  \end{equation}
  Contracting with $u^a$ gives
  \begin{equation}
    \label{eq:contract-with-na}
    n^au_a = W \underbrace{n^a n_a}_{=-1} + W \underbrace{n^av_a}_{=0} =- W \rightarrow \boxed{W = -n^av_a}.
  \end{equation}
  From Eq. (\ref{eq:3+1-decomposition}) we have
  \begin{equation}
    \label{eq:ua}
    u^a = W(n^a + v^a) = W(\alpha^{-1}(1, -\beta^i) + (0, v^i)) \rightarrow \boxed{u^a = \frac{W}{\alpha}(1, \alpha v^i - \beta^i)}.
  \end{equation}
  From $n^av_a = 0$ we have $v_0 = \beta^i v_i$. Thus,
  \begin{equation}
    \label{eq:u_a}
    u_a = W(n_a + v_a) = W(\alpha(-1,0,0,0) + (\beta^j v_j, v_i)) \rightarrow\boxed{u_a = W (-\alpha + \beta^j v_j, v_i)}.
  \end{equation}
  Using Eq. (\ref{eq:ua}) and Eq. (\ref{eq:u_a}) in the normalization condition $u^au_a = -1$ gives
  \begin{equation}
    \label{eq:Lorenzt-factor}
    \boxed{W = \frac{1}{\sqrt{1 - \gamma_{ij}v^i v^j}}}.
  \end{equation}
\item Lets start with Eq. \eqref{eq:identity-2} with $W^b = e^b_{(m)}$. This would give
  \begin{equation}
    \label{eq:conserved-variable}
    \textrm{l.h.s} = \partial_t(\sqrt{-g}T^t_b e^b_{(m)}) + \partial_k(\sqrt{-g}T^k_b e^b_{(m)}).
  \end{equation}
  So the conserved variable corresponding to $m = 0$ is $\sqrt{-g}T^t_b e^b_{(0)}$ and the flux is $\sqrt{-g}T^k_b e^b_{(0)}$. Now we choose the tetrad and coordinates so that $\mathbf{e} = \{\mathbf{n}, \partial_j\}$. So $e_{(0)} = \mathbf{n}$ and $e_{(j)} = \partial_j$ or $e^a_{(0)} = n^a = g^{ab}n_b = g^{ab}(-\alpha \delta^{0}_b) = -\alpha g^{a0}$. Therefore,
  \begin{equation}
    \label{eq:conserved-variable-t}
    \sqrt{-g}T^t_b e^b_{(0)} = \sqrt{-g}T^t_b (-\alpha)g^{b0} = -\alpha \sqrt{-g}(T^t_t g^{tt} + T^t_i g^{0i})
  \end{equation}
  Now, we can write the metric elements in terms the lapse ($\alpha$) and the shift ($\beta^i$). The vector along which time evolves, i.e., the time-coordinate basis vector is decomposed into it's component along the unit normal to the hypersurface ($n^a$) and it's component tangent to the hypersurface ($e^a$). $e^a$ could be further written in terms of the basis vectors on the hypersurface $e^a = \beta^i e^a_i$. So $t^\alpha = \alpha n^\alpha + \beta^i e^\alpha_i$. Therefore we can write
  \begin{equation}
    \label{eq:dxalpha}
    dx^a = t^a dt + e^a_i dx^i = n^a \alpha dt + e^a_i (\beta^i dt + dx^i)
  \end{equation}
  which implies
  \begin{equation}
    \label{eq:ds2}
    ds^2  = g_{ab}dx^a dx^b = \alpha^2\underbrace{g_{ab}n^a n^b}_{=-1} dt^2 +  \underbrace{g_{ab}e^a_i e^b_j}_{=\gamma_{ij}}(\beta^{i}\beta^{j}dt^2 + 2\beta^idtdx^j + dx^i dx^j).
  \end{equation}
  or
  \begin{equation}
    \label{eq:line-element}
    \boxed{ds^2 = -(\alpha^2 - \beta_i \beta^i)dt^2 + 2\beta_idx^i dt + \gamma_{ij} dx^i dx^j.}
  \end{equation}
  So,
  \begin{equation}
    \label{eq:metric-in-alpha-beta}
    \boxed{g_{ab} =
    \begin{pmatrix}
      -\alpha^2 + \beta_i \beta^i & \beta_i \\
      \beta_i & \gamma_{ij}
    \end{pmatrix}}, \qquad g^{ab} =
    \begin{pmatrix}
      -{1/\alpha^2} & {\beta^i}/{\alpha^2}\\
      \beta^i/\alpha^2 & \gamma^{ij} - \beta^i\beta^j/\alpha^2
    \end{pmatrix}.
  \end{equation}
  Now, $ -1/\alpha^2 = g^{tt} = \textrm{cofactor}(g_{tt})/\det{g_{ab}} = \det(\gamma_{ij})/g$, and therefore $\boxed{\sqrt{-g} = \alpha \sqrt{\gamma}}$.
  
  From Eq. \eqref{eq:stress-energy-perfect-fluid}, we have $T^a_b = \rho h u^a u_b + p \delta^a_b$. So, using Eq. \eqref{eq:ua} and \eqref{eq:u_a} we have
  \begin{equation}
    \label{eq:Ttt}
    T^t_t = \rho h u^t u_t + p =  \rho h \frac{W}{\alpha}W(-\alpha + \beta^j v_j) + p = \rho h \frac{W^2}{\alpha}(-\alpha + \beta^j v_j) + p
  \end{equation}
  and
  \begin{equation}
    \label{eq:Tti}
    T^t_i = \rho h u^t u_i = \rho h \frac{W}{\alpha}v_i
  \end{equation}
  Thus Eq. \eqref{eq:conserved-variable-t} becomes,
  \begin{align}
    \label{eq:conserved-variable-tt}
    \sqrt{-g}T^t_b e^b_{(0)}
    & = -\alpha \sqrt{-g}\left(-\frac{1}{\alpha^2}\right)\left(\rho h \frac{W^2}{\alpha}(-\alpha + \beta^j v_j) + p - \beta^i\rho h \frac{W}{\alpha}v_i\right)\\
    & = \alpha^{-1}\sqrt{-g}(- \rho h W^2 + p)\\
    & = \boxed{- \sqrt{\gamma}E}.
  \end{align}

  Lets now compute the flux in the $k$ direction for $m=0$ which we showed from Eq. \eqref{eq:conserved-variable} to be given by $\sqrt{-g}T^k_b e^b_{(0)}$ which for $e^b_{(0)} = n^b$ becomes
  \begin{equation}
    \label{eq:flux-tj}
    \sqrt{-g}T^k_b e^b_{(0)} = -\alpha \sqrt{-g} T^k_b g^{b0} = -\alpha\sqrt{-g}(T^k_t g^{tt} + T^k_i g^{it})
  \end{equation}
  Now,
  \begin{equation}
    \label{eq:Tkt-Tki}
    \begin{aligned}
      & T^k_t = \rho h u^k u_t = \rho h \frac{W}{\alpha}(\alpha v^k - \beta^k)W(-\alpha + \beta^j v_j) = \frac{\rho h W^2}{\alpha}(-\alpha^2 v^k + \alpha v^k \beta^j v_j + \alpha \beta^k - \beta^k v_j \beta^j)\\
      & T^k_i = \rho h u^k u_i + p \delta^k_i = \rho h \frac{W}{\alpha}(\alpha v^k - \beta^k)W v_i + p \delta^k_i = \frac{\rho h W^2}{\alpha}(\alpha v^k v_i - \beta^k v_i) + p \delta^k_i
  \end{aligned}
\end{equation}
Using above equations in Eq. \eqref{eq:flux-tj} with $g^{tt} = -1/\alpha^2, g^{ti} = \beta^i/\alpha^2$ gives

\begin{align}
  \label{eq:flux-tj-first-form}
  \sqrt{-g}T^k_b e^b_{(0)}
  & = - \frac{\sqrt{-g}}{\alpha}(\rho h W^2 (\alpha v^k - \beta^k) + \beta^k p) \\
  & = - \sqrt{\gamma}((E+p)(\alpha v^k - \beta^k) + \beta^k p)\\
  & =\boxed{-\sqrt{-g}\left[E\left(v^k - \frac{\beta^k}{\alpha}\right) + p v^k\right]}
\end{align}
Alternatively it could be written in the following form also,
\begin{align}
  \label{eq:flux-tj-final}
  \sqrt{-g}T^k_b e^b_{(0)}
  & = - \frac{\sqrt{-g}}{\alpha}(\rho h W^2 (\alpha v^k - \beta^k) + \beta^k p)\\
  & = - \sqrt{\gamma}(\beta^k(-\rho h W^2 + p) + \alpha \rho h W^2 v^k)\\
  & = \boxed{- \sqrt{\gamma}(\alpha S^k - \beta^k E)},
\end{align}
where $S^k = \rho h W^2 v^k$.

Now let us compute the conserved variable and the flux along $k$ direction for $m = j$. Again from Eq. \eqref{eq:conserved-variable} we find (with $m=j$) the conserved variable to be $\sqrt{-g}T^t_b e^b_{(j)} = \sqrt{-g}T^t_b \delta^b_{(j)} = \sqrt{-g}T^t_j$. Therefore using Eq. \eqref{eq:Tti} we find
\begin{equation}
  \label{eq:Conserved-variable-tj}
  \boxed{\sqrt{-g}T^t_b e^b_{(j)} = (\sqrt{-g}/\alpha)\rho h W^2 v_j = \sqrt{\gamma} S_j}
\end{equation}
Similarly the flux along $k$ is $\sqrt{-g}T^k_b e^b_{(j)} = \sqrt{-g}T^k_b \delta^b_j = \sqrt{-g}T^k_j$. Therefore, using Eq. \eqref{eq:Tkt-Tki} we get

\begin{equation}
  \label{eq:flux-kj-first-form}
  \boxed{\sqrt{-g}T^k_b e^b_{(j)} = \sqrt{-g}\left[\frac{\rho h W^2}{\alpha}(\alpha v^k v_j - \beta^k v_j) + p \delta^k_j\right] = \sqrt{-g}\left[S_j\left(v^k - \frac{\beta^k}{\alpha}\right) + p \delta^k_j\right]},
\end{equation}
Alternatively we can write in following form,
\begin{equation}
  \label{eq:flux-kj}
  \boxed{\sqrt{-g}T^k_b e^b_{(j)} = \sqrt{-g}\left[\rho h W^2 v^k v_j + p \delta^k_j - \rho h W^2 v_j \beta^k /\alpha\right] = \sqrt{\gamma}(\alpha S^k_j - S_j \beta^k)},
\end{equation}
where $S^{ij} = \rho h W^2 v^i v^j + p \gamma^{ij}$.
\item The mass conservation equation or the so called continuity equation could be written as
  \begin{align}
    \label{eq:mass-conservation}
    & \nabla_a(\rho u^a) = 0 \\
    & \Rightarrow \partial_a(\sqrt{-g}\rho u^a) = 0 \\
    & \Rightarrow \partial_t(\sqrt{-g}\rho u^t) + \partial_k (\sqrt{-g}\rho u^k) = 0\\
    & \Rightarrow \partial_t(\sqrt{-g}\rho\frac{W}{\alpha}) + \partial_k\left[\sqrt{-g}\rho W \left(v^k - \frac{\beta^k}{\alpha}\right)\right]=0\\
    & \Rightarrow \boxed{\partial_t(\sqrt{\gamma} D) + \partial_k\left[\sqrt{-g} D \left(v^k -\frac{\beta^k}{\alpha}\right)\right] = 0},
  \end{align}
  where $D = \rho W$.

  Using the results of the previous questions we can finally cast the conservative expressions of the relativistic-hydrodynamic equations in the compact form
  \begin{equation}
    \label{eq:conservative-expressions-relativistic-hydrodynamics}
    \partial_t(\sqrt{\gamma} \mathbf{q})+\partial_k (\sqrt{\gamma} \mathbf{f}^k(\mathbf{q})) = s(\mathbf{q})
  \end{equation}
  where
  \begin{equation}
    \label{eq:conservative-variables-and-fluxes}
    \mathbf{q} :=
    \begin{pmatrix}
      D \\
      S_j\\
      E
    \end{pmatrix},\qquad
    \mathbf{f}^k := \begin{pmatrix}
      D \left(v^k -\frac{\beta^k}{\alpha}\right)\\
      S_j\left(v^k - \frac{\beta^k}{\alpha}\right) + p \delta^k_j\\
      E\left(v^k - \frac{\beta^k}{\alpha}\right) + p v^k
    \end{pmatrix}\equiv
    \begin{pmatrix}
      D \left(v^k -\frac{\beta^k}{\alpha}\right)\\
      S^k_j - S_j \frac{\beta^k}{\alpha}\\
      S^k - E \frac{\beta^k}{\alpha}
    \end{pmatrix}
  \end{equation}

\item We can set $\alpha = 1, \beta^i = 0$ to make $ds^2$ in Eq. \eqref{eq:ds2} Minkowskian. From Eq. \eqref{eq:conservative-variables-and-fluxes} we immediately get
  \begin{equation}
    \label{eq:flux_minkowskian}
    \mathbf{f}^k =
    \begin{pmatrix}
      D v^k\\
      S_jv^k + p \delta^k_j\\
      (E + p)v^k
    \end{pmatrix}
  \end{equation}
  and Eq. \eqref{eq:conservative-expressions-relativistic-hydrodynamics} becomes (in Cartesian coordinate the $\Gamma$s are zero for Minkowski spacetime and hence the source term vanish)
  \begin{equation}
    \label{eq:conservative-expressions-minkowskian}
    \partial_t
    \boxed{\begin{pmatrix}
      D\\
      S_j\\
      E
    \end{pmatrix} + \partial_k
    \begin{pmatrix}
      D v^k\\
      S_jv^k + p \delta^k_j\\
      (E + p)v^k
    \end{pmatrix} = 0}
  \end{equation}
  where we have used $\gamma = 1$ for Minkowskian spacetime.

  \underline{In slow motion approximation ($v\ll 1, \epsilon + p/\rho_0 \ll 1$)}
  \begin{equation}
    \label{eq:D-in-slow-motion}
    W = \frac{1}{\sqrt{1 - v_i v^i}} \approx 1 + \frac{1}{2}v_i v^i, W^2\approx 1 + v_i v^i, h \approx 1
  \end{equation}
  Therefore we can make the following approximation,
  \begin{equation}
    \label{eq:slow-motion-minkowskian}
    \begin{aligned}
      & D = \rho_0 W \approx \rho_0 \\
      & Dv^k = \rho_0 W v^k \approx \rho_0 v^k\\
      & S_j = \rho_0 h W^2 v_j \approx \rho_0 v_j \\
      & S_jv^k + p \delta^k_j = \rho_0 h W^2 v_j v^k + p \delta^k_j \approx \rho_0 v_j v^k + p\delta^k_j\\
      & E = \rho_0 h W^2 - p \approx \rho_0 (1 + \epsilon + p/\rho_0)(1 + v_i v^i) - p \approx \rho_0(1 + \epsilon + v_i v^i)
    \end{aligned}
  \end{equation}
  Thus in slow approximation limit Eq. \eqref{eq:conservative-expressions-minkowskian} becomes
  \begin{equation}
    \label{eq:slow-motion-conservative-expressions-1}
    \partial_t
    \begin{pmatrix}
      \rho_0\\
      \rho_0 v_j \\
      \rho_0(1 + \epsilon + v_iv^i)
    \end{pmatrix} + \partial_k
    \begin{pmatrix}
      \rho_0 v^k \\
      \rho_0 v_j v^k + p\delta^k_j\\
      (\rho_0(1 + \epsilon + v_i v^i) + p)v^k
    \end{pmatrix} = 0
  \end{equation}
  As we can see, the Newtonian continuity and momentum conservation equation are recovered but the energy equation is not recovered correctly.

\item So lets go back to our SR equation in Eq. \eqref{eq:conservative-expressions-minkowskian} define $\tau = E - D$. Then We can have an equation of $\tau$ instead of $E$ and we can write conservative expressions as
  \begin{equation}
    \label{eq:conservative-expressions-in-tau}
    \partial_t
    \begin{pmatrix}
      D\\
      S_j\\
      \tau
    \end{pmatrix} + \partial_k
    \begin{pmatrix}
      D v^k\\
      S_jv^k + p \delta^k_j\\
      (\tau + p)v^k
    \end{pmatrix} = 0
  \end{equation}
  In slow motion limit $\tau \approx \rho_0(1 + \epsilon + v_i v^i) - \rho_0 (1 + (1/2)v_iv^i) \approx \rho_0(\epsilon + (1/2)v_i v^i)$ and Eq. \eqref{eq:conservative-expressions-in-tau} becomes
    \begin{equation}
    \label{eq:conservative-expressions-in-tau-slow-motion-limit}
    \boxed{\partial_t
    \begin{pmatrix}
      \rho_0\\
      \rho_0 v_j \\
      \rho_0(\epsilon + \frac{1}{2}v_i v^i)
    \end{pmatrix} + \partial_k
    \begin{pmatrix}
      \rho_0 v^k \\
      \rho_0 v_j v^k + p\delta^k_j\\
      (\rho_0(\epsilon + \frac{1}{2}v_i v^i) + p)v^k
    \end{pmatrix} = 0}
  \end{equation}
  and we recover the correct set of equations in the leading order.
\end{enumerate}


\subsection{Action principles}
\label{sec:action-principles}
The flux four-vector is defined as $n^a = n u^a$ where $u^a$ is the four-velocity. $n$ is related to the specific rest mass density as $\rho_0 = m_b n$ where $m_b$ is the mass per particle.
\begin{enumerate}
\item Let the action be given by $S = \int d^4x\sqrt{-g} \Lambda (n)$ where $\Lambda(n)$ is function of the number density $n$ only. So $\delta S = \int d^4x\sqrt{-g}\delta \Lambda = \int d^4x \sqrt{-g} (\partial \Lambda/\partial n^a)\delta n^a$ with the spacetime fixed. Thus the $\delta S = 0$ implies $\boxed{\mu_a := \partial \Lambda/\partial n^a = 0}$.

\item To ensure that the number flux is conserved we use $\nabla_a n^a = 0$, Thus we modify the Lagrangian to accomodate this constraint using Lagrange multiplier
  \begin{equation}
    \label{eq:Lagrangian-with-Lagrange-multiplier}
    S = \int d^4x \sqrt{-g}\left[\Lambda(n) + \lambda \nabla_a n^a\right]
  \end{equation}
  Now we vary the action
  \begin{align}
    \label{eq:vary-action-Lagrange-multiplier}
    \delta S
    & = \int d^4x \sqrt{-g}\left[\frac{\partial \Lambda}{\partial n^a}\delta n^a + \delta \lambda \nabla_a n^a + \lambda \nabla_a \delta n^a\right]\\
    & = \int d^4 x\sqrt{-g} \left[\frac{\partial \Lambda}{\partial n^a}\delta n^a + \nabla_a(\lambda \delta n^a) - \nabla_a \lambda \delta n^a + \nabla_a n^a\delta \lambda\right]\\
    & = \int d^4x \sqrt{-g} \left[\left(\mu_a - \nabla_a\lambda\right)\delta n^a + \nabla_a n^a \delta \lambda + \nabla_a(\lambda \delta n^a)\right]
  \end{align}
  The last term is a four-divergence and  goes away after integration and we have the set of e.o.m
  \begin{equation}
    \label{eq:set-of-eom}
    \begin{aligned}
      & \mu_a - \partial_a \lambda = 0\\
      & \nabla_a n^a = 0
    \end{aligned}
  \end{equation}
  The last one basically reproduces the number flux conservation. By operating covariant derivative on the first equation gives $\nabla_b \mu_a - \nabla_b\partial_a\lambda = 0$. Interchanging $a\leftrightarrow b$ and substracting gives
  \begin{equation}
    \label{eq:irrotationality}
    \nabla_a\mu_b - \nabla_b \mu_a = 0 \Rightarrow \boxed{\omega_{ab} = \nabla_{[a}\mu_{b]} = 0}
  \end{equation}
  $\omega_{ab}$ is the vorticity and fluid which obeys $\omega_{ab} = 0$ is called irrotational fluid.

\item However, we want to impose that the variaons act to perturb the flow lines normal to themselves. This imposes that
  \begin{equation}
    \label{eq:delta-na}
    \delta n^a = -\mathcal{L}_{\xi} n^a - n^a \nabla_b \xi^b = - \xi^b \nabla_b n^a + n^b\nabla_b \xi^a - n^a \nabla_b \xi^b = - \nabla_b(n^a \xi^b) + \nabla_b(n^b \xi^a)
  \end{equation}
  where we have used $\nabla_a n^a = 0$. Now let rewrite the variation of action $S = \int d^4x \sqrt{-g} \Lambda$ in terms of $\xi^a$. We have
  \begin{align}
    \label{eq:variation-of-action-with-xi}
    \delta S
    & = \int d^4x \sqrt{-g} \mu_a \delta n^a \\
    & = \int d^4x \sqrt{-g}\left[\mu_a \left(- \nabla_b(n^a \xi^b) + \nabla_b(n^b \xi^a)\right)\right]\\
    & = \int d^4x \sqrt{-g} \left[-\nabla_b(\mu_a n^a \xi^b) + \xi^b n^a \nabla_b \mu_a + \nabla_b(\mu_a n^b \xi^a) - n^b\xi^a \nabla_b \mu_a\right]\\
    & = \int d^4x \sqrt{-g} \left[\nabla_b(\mu_a(n^b\xi^a - n^a\xi^b)) + \xi^b n^a(\nabla_b \mu_a-\nabla_a\mu_b)\right]\\
    & \int d^4x \sqrt{-g} \xi^b n^a \omega_{ba} + \textrm{Boundary terms}
  \end{align}
  So we get the e.o.m to be
  \begin{equation}
    \label{eq:eom-single}
    \boxed{f_b := n^a\omega_{ab} = 0}
  \end{equation}

\item Equating $\Lambda = -\rho$, (where $\rho (n)$ is the energy density) we have $\delta \rho = - \mu_a \delta n^a$. Now we use the relation between the enthalpy, energy and pressure which is given by $n \mu = (\rho + p)$. Taking differential of both sides give $\boxed{\delta p = - n^a \delta \mu_a}$.

\item Now $\Lambda$ is function of $n_x^2, n_{xy}$ with $x={1,2}$ where $n_x$ is the number density of species $x$ and $n^2_{12} = - g_{ab}n^a_1 n^b_2$. Variation of $\Lambda$ would be given by
  \begin{align}
    \label{eq:variation-lambda-two-fluid}
    \delta \Lambda
    &  = \frac{\partial \Lambda}{\partial n_1^2}\delta n_1^2 + \frac{\partial \Lambda}{\partial n_2^2}\delta n_2^2 + \frac{\partial \Lambda}{\partial n_{12}^2}\delta n_{12}^2\\
    & = \left(-2 \frac{\partial \Lambda}{\partial n_1^2}\right)g_{ab}n_1^a\delta n_1^b
      + \left(-2 \frac{\partial \Lambda}{\partial n_2^2}\right)g_{ab}n_2^a\delta n_2^b
      + \left(-\frac{\partial \Lambda}{\partial n_{12}^2}\right)(g_{ab}n_1^a\delta n_2^b + g_{ab}\delta n_1^b n_2^a)\\
    & = \mathcal{B}^1g_{ab}n_1^a \delta n_1^b + \mathcal{B}^2g_{ab}n_2^a \delta n_2^b
      + \mathcal{A}^{12}(g_{ab}n_1^a\delta n_2^b + g_{ab}\delta n_1^b n_2^a),\quad \mathcal{B}^x := -2 \frac{\partial\Lambda}{\partial n_x^2},\mathcal{A}^{xy} = -\frac{\partial\Lambda}{\partial n_{xy}^2}\\
    & = \left(\mathcal{B}^1 n^a_1 + \mathcal{A}^{12}n_2^a\right)g_{ab}\delta n^b_1
      + \left(\mathcal{B}^2n_2^a + \mathcal{A}^{12}n_1^a\right)g_{ab}\delta n_2^b\\
    & = \sum_{x} \left(\mathcal{B}^x n^a_x + \mathcal{A}^{xy}n_y^a\right)g_{ab}\delta n^b_x\\
    & = \sum_{x} \mu_a^x \delta n_x^a,\qquad \textrm{where}\quad \mu_b^x:= \left(\mathcal{B}^x n^a_x + \mathcal{A}^{xy}n_y^a\right)g_{ab}
    \label{eq:conjugate-momentum-two-fluid}\\
    & = \sum_{x} \mu^x_a \left[\nabla_b\left(n_x^b \xi^a_x - n^a_x \xi^b_x\right)\right]\qquad \textrm{using Eq \eqref{eq:delta-na}}\\
    & = \sum_{x} \nabla_b\left[\mu^x_a\left(n_x^b \xi^a_x - n^a_x \xi^b_x\right)\right] + \left(n_x^b \xi^a_x - n^a_x \xi^b_x\right) \nabla_b\mu^x_a \\
    & = \sum_{x} \nabla_b\left[\mu^x_a\left(n_x^b \xi^a_x - n^a_x \xi^b_x\right)\right] + n^a_x \left(\nabla_b \mu^x_a - \nabla_a \mu^x_b\right)\xi^b_x\\
    & = \sum_{x} \nabla_b\left[\mu^x_a\left(n_x^b \xi^a_x - n^a_x \xi^b_x\right)\right] - n_x^a \omega^x_{ab} \xi^b_x
  \end{align}

  So the variation of action becomes
  \begin{equation}
    \label{eq:variaons-action-two-fluid}
    \delta S = \int d^4x \sqrt{-g} \delta \Lambda = \textrm{B.T} - \int d^4x \sqrt{-g}\sum_{x} f_b^x \xi^b_x
  \end{equation}
So the equations of motion are given by $\boxed{f^x_b = n^a_x\omega^x_{ab} = 0}$.
\item Conjugate momentum $\mu_a^x$ of $n^x_a$ is given by Eq. \eqref{eq:conjugate-momentum-two-fluid}, $\mu_a^x = g_{ab}(\mathcal{B}^x n_x^b + \mathcal{A}^{xy}n^b_y)$. So the conjugate momentum of species $x$ is only parallel to its number flux if
  \begin{align}
    \label{eq:condition-for-parallel-to-its-number-flux}
    \mathcal{A}^{xy}
    & = -\frac{\partial \Lambda}{\partial n^2_{xy}} = 0 
  \end{align}
  or, $\boxed{\frac{\partial \Lambda}{\partial n^a_1 \partial n_2^b} = 0}$.
\end{enumerate}

\end{document}